\documentclass[11pt]{article}
\usepackage[margin=1in]{geometry}
\usepackage[english]{babel}
% \usepackage{listingstyle}
\usepackage{graphicx}
\usepackage{caption}
\usepackage{subcaption}
% \usepackage{coz}
\usepackage{url}
\usepackage{array}
\usepackage{multirow}
% \usepackage[autolanguage]{numprint}


\title{\textbf{CS5232 Project Proposal\\} }
\author{Akanksha Tiwari (A0123476E) \\
Antoine Francois Pascal Creux (A0123427M)\\
Ashish Dandekar (A0123873A)}
\date{March 18th, 2014}

\begin{document}
\maketitle


\section{Concept Study}
Goodreads is a website launched in early 2007, which lets ``people find and share the books they like and improve the process of reading and learning throughout the world.'' It is the world's largest site for readers and book recommendations with a user base of about  30 million members along with 34 million reviews from 900 million books as recorded in 2015 \cite{goodreads:aboutus}.

Goodreads is a website launched in early 2007, which lets ``people find and share the books they like and improve the process of reading and learning throughout the world.'' It is the world’s largest site for readers and book recommendations with a user base of about  30 million members along with 34 million reviews from 900 million books as recorded in 2015 \cite{goodreads:aboutus}.

Goodreads provides a multitude of features to its users. It goes beyond the traditional rating and reviewing of books by allowing users to make friends and join and form reading groups based on their literary tastes. Users can not only see what their friends have read, but they can also meet new people with similar reading interests. They can make recommendations to friends , follow authors, track the books they are currently reading, have read and want to read. In addition, goodreads provides personalized recommendations to book readers by analyzing the user data. \\

To capture the notion of similar interest among the users of goodreads  the concept of one-mode analyses of bipartite network data(also known as affiliation network in sociology terminology) can be used \cite{wasserman}. Such analyses use matrices derived from the affiliation matrix, $\mathcal{A}$, wherein actors are represented in rows and the events are represented in columns.

Co-membership of actors $i$ and $j$ for an event $k$ is identified if the rows $a_i$ and $a_j$ both have 1 in the column $k$ . Hence,  $a_{ik} = 1$ and $a_{jk} = 1$ indicates that both actors participated in the event $k$. So, the total number of co-memberships for the two actors can be computed from the number of times that $a_{ik} = 1$ and $a_{jk} = 1$ where $k$ takes all possible values. The number of events with which both actors $i$ and $j$ are associated will vary from  0(if both have no affiliated events in common) to $h$(if both have all affiliated events in common).\\

Community detection is widely discussed topic in the social networking literature\footnote{Social networks are generally sparse networks. When the networks are dense, the community detection does not seem to be a wise option. The dense networks possess large number of inter-community edges which yield poor partitioning. For dense networks, discrete data analysis techniques are used.}\cite{clauset}. There is no universally accepted definition of community since the idea of community is pertinent to the underlying network. The general picture of community detection algorithm relies on the quality function which defines the quality of the cluster. The algorithm aims at either maximizing or minimizing(depends on the way quality function has been defined) cumulative quality of the network. Generally the quality function for a community is defined to be the difference between intra-community edges and inter-community edges.

There are different classes of community detection algorithms. Graph partitioning algorithms return two partitions of the network using techniques like max-flow min-cut, conductance, etc. Despite their scalability, these algorithms require the analyst to have some idea about the groups in the network beforehand. So they are not good for the exploration of the groups in the community. Similarly partition clustering algorithms, like k-means clustering, needs analyst conjecture about the existence of clusters. These algorithms also require the graph to be embedded in the metric space to quantify the notion of the distance. Spectral clustering algorithms reuqire eigenvalue decomposition of the network graph which is prohibitive for massive networks. Hierarchical partitioning techniques, comprising of agglomerative and divisive partitioning methods, are good for the exploratory analysis. Agglomerative clustering techniques are computationally expensive. Divisive algorithms are top-down partitioning techniques which goes on deleting edges one by one which causes formation of communities. Girvan-Newman algorithm is one of the popular divisive community detection algorithm which we poropose to use for detecting user communities in goodreads.


\section{Problem Description}
In this project, we aim to probe the groups of users within goodreads based on their reading interests. As stated earlier, goodreads is more than just a book reviewing website. Users and authors on goodreads can connect amongst themselves forming a rich social network. We do not want to observe the groups/clusters within such a social network, which already exists in goodreads, since this network may not necessarily capture the reading interests of the users in it. We will use the information about the books read by the users and ratings given by them to capture the ties amongst the users.\\

In this project, we aim to probe the groups of users within goodreads based on their reading interests. As stated earlier, goodreads is more than just a book reviewing website. Users and authors on goodreads can connect amongst themselves forming a rich social network. We do not want to observe the groups/clusters within such a social network, which already exists in goodreads, since this network may not necessarily capture the reading interests of the users in it. We will use the information about the books read, the authors people are fans of and the group memberships to capture the ties amongst the users. \\\\

In this section, we only focus on the book read by two users. Let $U$ and $B$ be the set of users and the set of books on goodreads respectively. We will construct an undirected bipartite graph $G(V, E)$ using these two sets where $V ~=~ U \cup B$. An edge exists betweem a vertex $u_i \in U$ and a vertex $b_k \in B$ if a user $u_i$ has read book $b_k$.\\\\
We consider the rating given by a user $u_i$ to a book $b_k$ to assign weight to the edge between them. The weight to this edge plays an important role in the semantics of the network. Consider a scenario wherein a book $b_k$ is read by both the users $u_i$ and $u_j$. Suppose user $u_i$ has liked the book and hence has positively rated the book whereas user $u_j$ has not liked the book and so has given a low rating to the book. If we do not take this fact into the account then these users might end up being placed in a same reading group. So as to prevent this situation, we assign weight to the edges in $G$. Let $r_{ik}$ be the rating given by user $u_i$ to book $b_k$. Let $w \rightarrow E \times \{1, -1\}$ be the weight function.\\
$\forall (i, k) \in E$, 
\[
	w((i, k)) = \left\{\def\arraystretch{1.2}%
		\begin{array}{@{}c@{\quad}l@{}}
		1 & r_{ik} \geq \alpha\\
		-1 & r_{ik} < \alpha\\
		\end{array}\right.
\]
where $\alpha$ is some threshold which can be tuned during the experimentation.\\
A similar study can be conducted based on the author $a_k \in A$ or the group membership $g_k \in G$, for instance:\\
$\forall (i, k) \in E=U \cup G$, 
\[
	w((i, k)) = \left\{\def\arraystretch{1.2}%
		\begin{array}{@{}c@{\quad}l@{}}
		1 & u_i \in g_k\\
		0 & u_i \notin g_k\\
		\end{array}\right.
\]
\\\\

Let $\mathcal{M}$ be a $|U| \times |B|$ matrix which represents the underlying graph $G$. If there exists an edge between user $u_i$ and book $b_k$ then $\mathcal{M}_{ik}^{th}$ entry in the matrix is set to the weight of the corresponding edge otherwise the entry is set to zero. We obtain $\mathcal{U}$, matrix encoding the relationship amongst the users, using $\mathcal{M}$ by calculating $\mathcal{M}\mathcal{M}^{T}$. The graph encoded by $\mathcal{U}$ is the graph of the interest for finding the user groups. We propose to explore the interesting user groups by using graph partitioning algorithms like Girvan-Newmann algorithm.


\section{Underlying Assumptions}

Goodreads data set is not available online and  we will have to acquire all the data through their APIs. Hence, as of now, it is not feasible to state appropriate assumptions since we do not have an exact picture of the data. Once we have obtained all the data we will update this section.

\section{Data acquisition}
Goodreads data is not available online, we have to acquire all data through their APIs. Stanford students have already studied the Goodreads platform by building a product recommendation system for it\cite{stanford:goodreads}. They acquired data through the \textit{book.shelves} api method. Given a \textit{user\_id} and a bookshelf name, they were able to query all the books a user may have put in this virtual bookshelf. The major issue is that goodreads does not list every user\_id. So, they decided to query for user\_id randomly to query the books people have put in their public ‘reading' shelves. They gathered data for 4000 users and their 'reading books'.

We have decided to query the data using other APIs, available at \cite{goodreads:api}:
\begin{itemize}
\item group.list - List groups for a given user
\item group.members - Return members of a particular group
\item group.show - Get info about a group
\item fanship.show - Show author fanship information
\item owned\_books.list - List books owned by a user
\item shelves.list - Get shelves owned by a user
\item user.show - Get shelves owned by a user
\item user.followers - Get user's followers
\item user.following - Get people a user is following
\item user.friends - Get people a user is following
\end{itemize}

Through these APIs we get information related to the network of a user such as :
\begin{itemize}
\item the users followed by a user and the users who follow him (similar to twitter)
\item the reading groups joined by a user
\item the authors who's fan the user is
\item the books stored in the shelves of a user
\item all the friends of a user
\end{itemize}

Goodreads has 30 million users which may be too large for our study, but the goal is to gather an acceptable amount of data. Goodreads provides the top users in certain categories such as the top 50 active users, top 50 readers, top 50 most popular reviewers as well as the best reviews. These statistics are computed for the last week, last month, last 12 months and all the time for each country( as well as worldwide).
Based on the top users, we will have a list of user\_id from which we can query the data. On the other hand, for each user, we will store the books read and stored on the shelves, the authors the user is a fan of and the groups the user belongs to.\
Also, we  can only query a user data if the profile is public. Based on our initial investigation, most profiles are public and thus this  should not be an issue. However, we will store the information that a user profile is private as well.

\section{Project Objectives}

Based on the large information we can have on a user, we can generate several kinds of network:
\begin{itemize}
\item a Twitter network based on the following/followers relation
\item a Group network, as same members of a group are linked
\item a Author network, as being fan of the same author strenghen the bound between two users
\item a Book network, as having read the same books show similar tastes in literature
\item a Friend network, already implemented in goodreads
\end{itemize}


% http://networkx.github.io/documentation/networkx-1.9.1/reference/generated/networkx.algorithms.cluster.clustering.html?highlight=clustering#networkx.algorithms.cluster.clustering

% http://jponnela.com/web_documents/a9.pdf     network x clustering algo

% Girvan-Newman algo
% https://en.wikipedia.org/wiki/Girvan%E2%80%93Newman_algorithm

% Graph tool algo
% http://graph-tool.skewed.de/static/doc/clustering.html

% Clustering Stanford
% http://web.stanford.edu/class/cs345a/slides/12-clustering.pdf

% Stanford goodreads
% http://cs229.stanford.edu/proj2008/IsaacsonSebastian-GoodReadsRecommendations.pdf


\section{Proposed Contributions}
As of the literature survey done upto now there has not been a study on the detection of communities of readers with similar reading interests on goodreads dataset. This work will aim to identify such communities using algorithms for community detection such as Girvan-Newman algorithm and compare the communities detected with the already formed groups on goodreads and analyse how different or similar they are. This would give us insights into whether the friendship and groups formed on goodreads are based on reading intersts or there are other factors that come into play.It be especially useful to authors in identifying the right target audience for their work.

In addition, if we get time we can also look at how the recommendations and friendships in the goodreads' user network affects their choice of books. This information can be used by gooddreads themselves to improve their recommendation engine.

\section{Success Measures}

Our entire work depends on acquiring the data in a short amount of time. We consider we validate this part if we manage to gather at least one million users with all their features (friends, books owned, followers, users following\dots) stored.\\
% Then, we will implement a clustering algorithm presented in section \ref{sec:literature_review} that we will try to apply.
Then, we will evaluate our clustering algorithm by studying by hand the communities formed. We expect to see communities gathered around common literature tastes such as science-fiction, romance or detective novels. 


\section{Project Plan}
\begin{itemize}
\item Week 0 - Identifying the project area, datasets to work on and conducting literature review.
\item Week 1 -  Collecting the data from the various goodreads APIs as mentioned above.
\item Week 2 -  Data cleaning, formulating assumptions based on the data and modifying the project objectives accordingly. These changes will be reflected in the Project Milestone report.
\item Week 3 \& Week 4 - Implementing the community detection algorithms on the data and analysing the results.
\item Week 5 - Interpretation and documentation of the results
\end{itemize}
\section{Deliverables}

The deliverables will be composed of a commented and tested code that will:
\begin{itemize}
\item Query the Goodreads API
\item Store the data in a database
\item Implement the community clustering
\item Apply the algorithm to goodreads data
\end{itemize}

Also, a deep study of the results of the algorithm will be conducted through comments and illustrations.

\newpage

\bibliographystyle{abbrv}
\bibliography{references}
\end{document}
