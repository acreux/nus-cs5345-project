\title{\textbf{\Huge Book-taste clustering in goodreads communities}}

\author{{\Large Akanksha Tiwari (A0123476E)} \\
	{\Large Antoine Francois Pascal Creux (A0123427M)}\\
	{\Large Ashish Dandekar (A0123873A)} \\\\
	National University of Singapore,\\
	Singapore
}
\date{\today}

\documentclass[11pt]{article}
\usepackage[margin=1in]{geometry}
\usepackage[english]{babel}
% \usepackage{listingstyle}
\usepackage{graphicx}
\usepackage{caption}
\usepackage{subcaption}
% \usepackage{coz}
\usepackage{url}
\usepackage{array}
\usepackage{multirow}
% \usepackage[autolanguage]{numprint}

\begin{document}
\maketitle

\newpage

\section{Intro / To modify}


Goodreads is a website launched in early 2007, which lets ``people find and share the books they like and improve the process of reading and learning throughout the world.'' It is the world's largest site for readers and book recommendations with a user base of about 30 million members along with 34 million reviews from 900 million books as recorded in 2015 \cite{goodreads:aboutus}.

Goodreads provides a multitude of features to its users. It goes beyond the traditional rating and reviewing of books by allowing users to make friends and join and form reading groups based on their literary tastes.
Users can not only see what their friends have read, but they can also meet new people with similar reading interests. They can make recommendations to friends, follow authors, track the books they are currently reading, have read and want to read. In addition, goodreads provides personalized recommendations to book readers by analyzing the user data. \\


In this project, we aim to probe the groups of users within goodreads based on their reading interests. As stated earlier, goodreads is more than just a book reviewing website. Users and authors on goodreads can connect amongst themselves forming a rich social network. We do not want to observe the groups/clusters within such a social network, which already exists in goodreads, since this network may not necessarily capture the reading interests of the users in it. We will use the information about the books read by the users and ratings given by them to capture the ties amongst the users.\\

\section{Scraping}

The first step in our study is getting the data from goodreads.com.

\bibliographystyle{abbrv}
\bibliography{references}
\end{document}
